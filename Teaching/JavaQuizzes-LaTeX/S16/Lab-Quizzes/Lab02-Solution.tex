\documentclass[addpoints]{exam}
\usepackage[utf8]{inputenc}
%%% Preamble for CS1800 Lab Quiz%%%

\usepackage[margin=1in]{geometry}
\geometry{letterpaper,textwidth=350pt,textheight=680pt,tmargin=60pt,
            left=72pt,footskip=24pt,headsep=18pt,headheight=14pt}
% Packages required to support encoding
\usepackage{ucs}
\usepackage[utf8x]{inputenc}
%%%%%%%%%%%%%%%%%%%
% To get dots between the entry name and the 
% page number in the Table Of Contents
\usepackage{tocloft}
\renewcommand\cftsecfont{\normalfont}
\renewcommand\cftsecpagefont{\normalfont}
\renewcommand{\cftsecleader}{\cftdotfill{\cftsecdotsep}}
\renewcommand\cftsecdotsep{\cftdot}
\renewcommand\cftsubsecdotsep{\cftdot}
%%%%%%%%%%%%%%%%%%%%
\usepackage{amsmath}
\usepackage{amssymb}
\usepackage[LGRgreek]{mathastext} % no italics in math mode
\usepackage{textcase}
\usepackage{soul}
\usepackage{booktabs} % for \midrule in eqnarray
\usepackage{indentfirst}
\usepackage{subfiles}
%%%%%%%%%%%%%%%%
\newif\ifdraft
%\drafttrue
\draftfalse
%%%%%%%%%%%%%%%%

\usepackage[pdftex]{graphicx}
\usepackage{hyperref}
\usepackage{xspace}
\usepackage[usenames,dvipsnames]{color}
\hypersetup{colorlinks=true,urlcolor=blue}
\usepackage{listings}
\definecolor{dkgreen}{rgb}{0,0.6,0}
\definecolor{gray}{rgb}{0.5,0.5,0.5}
\definecolor{mauve}{rgb}{0.58,0,0.82}

%%%%%%%%%%%%%%%%%%%%%%%%%%
% New command for inserting names in the document
\newcommand{\name}[1]{\textcolor{red}{\textbf{[#1]}}}
% New commands for inserting points in the document
\newcommand{\pt}[1]{\textbf{[#1 point]}}
\newcommand{\pts}[1]{\textbf{[#1 points]}}

\lstset{%frame=tb, % to form a frame around code
  language=Java,
  aboveskip=3mm,
  belowskip=3mm,
  showstringspaces=false,
  columns=flexible,
  basicstyle={\ttfamily}, % change to \small\ttfamily 
  numbers=none,
  numberstyle=\large\color{black},
  keywordstyle=\color{blue},
  commentstyle=\color{dkgreen},
  stringstyle=\color{mauve},
  breaklines=true,
  breakatwhitespace=true,
  tabsize=3
}
\usepackage[framemethod=TikZ]{mdframed}
\mdfdefinestyle{MyFrame}{%
    linecolor=black,
    outerlinewidth=1pt,
    roundcorner=20pt,
    innertopmargin=\baselineskip,
    innerbottommargin=\baselineskip,
    innerrightmargin=20pt,
    innerleftmargin=20pt,
    backgroundcolor=white!50!white}
%%%%%%%%%%%%%%%%

%% New Commands %%
\newcommand{\Ans}{\textcolor{red}{ [Answer] }}
\newcommand{\tf}{ (T / F) }
\newcommand{\Instructions}{Complete the following quiz questions and return the sheet to your TA. Do this at the start of the lab session.}
%%%%

\sodef\allcapsspacing{\upshape}{0.15em}{0.65em}{0.6em}%

\makeatletter
\def\maketitle{%
\par{\parskip 0pt
\hrule height 0.75pt\vspace{1ex}
\par\noindent
\begin{minipage}{0.6\textwidth}
\scshape
Purdue University -- CS 18000 \\
Problem Solving And \\ Object-Oriented Programming \end{minipage}%
\begin{minipage}{0.4\textwidth}
\raggedleft
\@title\\[0.2ex]
\textit{Name, Lab Sec: \hrulefill}\\[0.2ex]
\textit{\@date} %\textit{April 08, 2015} % 
\end{minipage}
\par\vspace{1ex}
\hrule height 1pt
\vspace{2ex}
\par
}}
\makeatother
\author{}
% auto generate a title
\AtBeginDocument{\maketitle}


\title{Lab 2 Quiz}
\date{Week 2}
\begin{document} 

\Instructions
\begin{enumerate}

\item \pts{6} Break down CS180 into 3 abstractions, each with 2 operations and 2 quantities. \newline
(e.g. \textit{Wheel} class has an operation \textif{rotate} with a quantity \textit{radius}).
\newline


\begin{center}
\begin{tabular}{ |c|c|c|c| } 
\hline
\textbf{Abstraction (Class)}& \textbf{Operation (Method)}& \textbf{Quantity(Variable)} \\
\hline
&   &   \\ 
\multirow{\textbf{e.g. Student}} & \textbf{e.g. Completes homework, attends lecture} & \textbf{e.g. Name, PUID} \\ 
&   &   \\ 
\hline
&   &   \\
TA & Answers questions, grades labs  & Name, classification  \\ 
&   &   \\ 
\hline
&   &   \\
Professor & Lectures material, posts grades  & Name, email  \\ 
&   &   \\ 
\hline
\end{tabular}
\end{center}
\Ans These are just examples.
\vspace{1em}


\item \pts{2} Describe an algorithm (i.e. a series of steps) to convert a negative binary number represented in \textbf{two's-complement} to its base 10 equivalent.
\newline
\Ans \newline
Flip the bits, add one, then convert to decimal.
\newline

\item \pts{2} Convert 126_10. Show your work.
\newline
\Ans \newline
126  /  2 = 63,  Rem: 0 \newline
63  /  2 = 31,  Rem: 1 \newline
31 / 2 = 15,  Rem: 1 \newline
15 / 2 = 7, Rem: 1 \newline
7 / 2 = 3, Rem: 1 \newline
3 / 2 = 1, Rem: 1 \newline
1 / 2 = 0, Rem: 1 \newline
\newline
= 1111110_2
\newline


\end{enumerate}
\end{document}
              