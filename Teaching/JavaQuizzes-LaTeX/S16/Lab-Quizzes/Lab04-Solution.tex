\documentclass[addpoints]{exam}
\usepackage[utf8]{inputenc}
%%% Preamble for CS1800 Lab Quiz%%%
\usepackage[margin=1in]{geometry}
\geometry{letterpaper,textwidth=350pt,textheight=680pt,tmargin=60pt,
            left=72pt,footskip=24pt,headsep=18pt,headheight=14pt}
% Packages required to support encoding
\usepackage{ucs}
\usepackage[utf8x]{inputenc}
%%%%%%%%%%%%%%%%%%%
% To get dots between the entry name and the 
% page number in the Table Of Contents
\usepackage{tocloft}
\renewcommand\cftsecfont{\normalfont}
\renewcommand\cftsecpagefont{\normalfont}
\renewcommand{\cftsecleader}{\cftdotfill{\cftsecdotsep}}
\renewcommand\cftsecdotsep{\cftdot}
\renewcommand\cftsubsecdotsep{\cftdot}
%%%%%%%%%%%%%%%%%%%%
\usepackage{amsmath}
\usepackage{amssymb}
\usepackage[LGRgreek]{mathastext} % no italics in math mode
\usepackage{textcase}
\usepackage{soul}
\usepackage{booktabs} % for \midrule in eqnarray
\usepackage{indentfirst}
\usepackage{subfiles}
%%%%%%%%%%%%%%%%
\newif\ifdraft
%\drafttrue
\draftfalse
%%%%%%%%%%%%%%%%

\usepackage[pdftex]{graphicx}
\usepackage{hyperref}
\usepackage{xspace}
\usepackage[usenames,dvipsnames]{color}
\hypersetup{colorlinks=true,urlcolor=blue}
\usepackage{listings}
\definecolor{dkgreen}{rgb}{0,0.6,0}
\definecolor{gray}{rgb}{0.5,0.5,0.5}
\definecolor{mauve}{rgb}{0.58,0,0.82}

%%%%%%%%%%%%%%%%%%%%%%%%%%
% New command for inserting names in the document
\newcommand{\name}[1]{\textcolor{red}{\textbf{[#1]}}}
% New commands for inserting points in the document
\newcommand{\pt}[1]{\textbf{[#1 point]}}
\newcommand{\pts}[1]{\textbf{[#1 points]}}

\lstset{%frame=tb, % to form a frame around code
  language=Java,
  aboveskip=3mm,
  belowskip=3mm,
  showstringspaces=false,
  columns=flexible,
  basicstyle={\ttfamily}, % change to \small\ttfamily 
  numbers=none,
  numberstyle=\large\color{black},
  keywordstyle=\color{blue},
  commentstyle=\color{dkgreen},
  stringstyle=\color{mauve},
  breaklines=true,
  breakatwhitespace=true,
  tabsize=3
}
\usepackage[framemethod=TikZ]{mdframed}
\mdfdefinestyle{MyFrame}{%
    linecolor=black,
    outerlinewidth=1pt,
    roundcorner=20pt,
    innertopmargin=\baselineskip,
    innerbottommargin=\baselineskip,
    innerrightmargin=20pt,
    innerleftmargin=20pt,
    backgroundcolor=white!50!white}
%%%%%%%%%%%%%%%%

%% New Commands %%
\newcommand{\Ans}{\textcolor{red}{ [Answer] }}
\newcommand{\tf}{ (T / F) }
\newcommand{\Instructions}{Complete the following quiz questions and return the sheet to your TA. Do this at the start of the lab session.}
%%%%

\sodef\allcapsspacing{\upshape}{0.15em}{0.65em}{0.6em}%

\makeatletter
\def\maketitle{%
\par{\parskip 0pt
\hrule height 0.75pt\vspace{1ex}
\par\noindent
\begin{minipage}{0.6\textwidth}
\scshape
Purdue University -- CS 18000 \\
Problem Solving And \\ Object-Oriented Programming \end{minipage}%
\begin{minipage}{0.4\textwidth}
\raggedleft
\@title\\[0.2ex]
\textit{Name, Lab Sec: \hrulefill}\\[0.2ex]
\textit{\@date} %\textit{April 08, 2015} % 
\end{minipage}
\par\vspace{1ex}
\hrule height 1pt
\vspace{2ex}
\par
}}
\makeatother
\author{}
% auto generate a title
\AtBeginDocument{\maketitle}


\title{Lab 4 Quiz}
\date{Week 4}
\begin{document} 

\Instructions
\begin{enumerate}
\newline
\item \pts{6}
Fill in the appropriate Java operator for each Math symbol.

\begin{oneparchoices} 
\hspace{0.2cm}

 \choice Equals:  \Ans == \vspace{1em} 
 \newline
 \choice Not equals: \Ans !=  \vspace{1em}
 \newline 

 \choice Less than: \Ans <  \vspace{1em}
 \newline 

 \choice Less than or equals: \Ans <=   \vspace{1em}
 \newline 

 \choice Greater than: \Ans >  \vspace{1em}
 \newline

 \choice Greater than or equals: \Ans >=
    \newline

 \end{oneparchoices}
 

 
\item \pts{2}
Fill in the appropriate bitwise operators.

\begin{oneparchoices} 
\hspace{0.2cm}
\vspace{2em}
 
 \choice NOT (complement): \Ans \~
 \newline
 \choice AND: \Ans  \& \vspace{1em}

 
 \end{oneparchoices}
 

 \item \pt{1} \tf
This code segment checks the condition of \textit{x} is equal to 4.
\begin{lstlisting}
if( x = 4)
    x++;
}

\end{lstlisting}
\Ans False. should be x == 4

\item \pt{1}
What is the output of this code segment?
\begin{lstlisting}
char grade = 'A';
switch( grade ) {
    case 'A': System.out.print("Hello ");
    case 'B': System.out.print("Student ");
    case 'C': System.out.print("Welcome ");
    case 'D': System.out.print("To ");
    default: System.out.println("CS180");
}

\end{lstlisting}
\begin{oneparchoices} 
\hspace{0.2cm}

 \choice Hello CS180 \vspace{1em}  
 \newline
 \choice \Ans Hello Student Welcome To CS180   \vspace{1em}
 \newline 

 \choice Hello Student   \vspace{1em}
 \newline 

 \choice Hello Welcome To CS180   \vspace{1em}
 \newline \newline

 \end{oneparchoices}
\end{enumerate}
\end{document}
              