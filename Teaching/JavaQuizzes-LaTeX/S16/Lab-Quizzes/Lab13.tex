\documentclass[addpoints]{exam}
\usepackage[utf8]{inputenc}
%%% Preamble for CS1800 Lab Quiz%%%
\usepackage[margin=1in]{geometry}
\geometry{letterpaper,textwidth=350pt,textheight=680pt,tmargin=60pt,
            left=72pt,footskip=24pt,headsep=18pt,headheight=14pt}
% Packages required to support encoding
\usepackage{ucs}
\usepackage[utf8x]{inputenc}
%%%%%%%%%%%%%%%%%%%
% To get dots between the entry name and the 
% page number in the Table Of Contents
\usepackage{tocloft}
\renewcommand\cftsecfont{\normalfont}
\renewcommand\cftsecpagefont{\normalfont}
\renewcommand{\cftsecleader}{\cftdotfill{\cftsecdotsep}}
\renewcommand\cftsecdotsep{\cftdot}
\renewcommand\cftsubsecdotsep{\cftdot}
%%%%%%%%%%%%%%%%%%%%
\usepackage{amsmath}
\usepackage{amssymb}
\usepackage[LGRgreek]{mathastext} % no italics in math mode
\usepackage{textcase}
\usepackage{soul}
\usepackage{booktabs} % for \midrule in eqnarray
\usepackage{indentfirst}
\usepackage{subfiles}
%%%%%%%%%%%%%%%%
\newif\ifdraft
%\drafttrue
\draftfalse
%%%%%%%%%%%%%%%%

\usepackage[pdftex]{graphicx}
\usepackage{hyperref}
\usepackage{xspace}
\usepackage[usenames,dvipsnames]{color}
\hypersetup{colorlinks=true,urlcolor=blue}
\usepackage{listings}
\definecolor{dkgreen}{rgb}{0,0.6,0}
\definecolor{gray}{rgb}{0.5,0.5,0.5}
\definecolor{mauve}{rgb}{0.58,0,0.82}

%%%%%%%%%%%%%%%%%%%%%%%%%%
% New command for inserting names in the document
\newcommand{\name}[1]{\textcolor{red}{\textbf{[#1]}}}
% New commands for inserting points in the document
\newcommand{\pt}[1]{\textbf{[#1 point]}}
\newcommand{\pts}[1]{\textbf{[#1 points]}}

\lstset{%frame=tb, % to form a frame around code
  language=Java,
  aboveskip=3mm,
  belowskip=3mm,
  showstringspaces=false,
  columns=flexible,
  basicstyle={\ttfamily}, % change to \small\ttfamily 
  numbers=none,
  numberstyle=\large\color{black},
  keywordstyle=\color{blue},
  commentstyle=\color{dkgreen},
  stringstyle=\color{mauve},
  breaklines=true,
  breakatwhitespace=true,
  tabsize=3
}
\usepackage[framemethod=TikZ]{mdframed}
\mdfdefinestyle{MyFrame}{%
    linecolor=black,
    outerlinewidth=1pt,
    roundcorner=20pt,
    innertopmargin=\baselineskip,
    innerbottommargin=\baselineskip,
    innerrightmargin=20pt,
    innerleftmargin=20pt,
    backgroundcolor=white!50!white}
%%%%%%%%%%%%%%%%

%% New Commands %%
\newcommand{\Ans}{\textcolor{red}{ [Answer] }}
\newcommand{\tf}{ (T / F) }
\newcommand{\Instructions}{Complete the following quiz questions and return the sheet to your TA. Do this at the start of the lab session.}
%%%%

\sodef\allcapsspacing{\upshape}{0.15em}{0.65em}{0.6em}%

\makeatletter
\def\maketitle{%
\par{\parskip 0pt
\hrule height 0.75pt\vspace{1ex}
\par\noindent
\begin{minipage}{0.6\textwidth}
\scshape
Purdue University -- CS 18000 \\
Problem Solving And \\ Object-Oriented Programming \end{minipage}%
\begin{minipage}{0.4\textwidth}
\raggedleft
\@title\\[0.2ex]
\textit{Name, Lab Sec: \hrulefill}\\[0.2ex]
\textit{\@date} %\textit{April 08, 2015} % 
\end{minipage}
\par\vspace{1ex}
\hrule height 1pt
\vspace{2ex}
\par
}}
\makeatother
\author{}
% auto generate a title
\AtBeginDocument{\maketitle}


\title{Lab 13 Quiz}
\date{Week 13}
\begin{document} 

\Instructions
\hspace{0.5em}Consider the following code for Q1 and Q2;
\begin{enumerate}
\begin{lstlisting}
abstract class Employee {
	private String name;
	public Employee(String name) {
		this.name = name;
	}
	public String getName() {
	    return name;
	}
	public abstract int getTenure();
}
class Salary extends Employee {
	private int tenure;
	public Salary(String name, int tenure) {
		super(name);
		this.tenure = tenure;
	}
	public int getTenure() {
	    return this.tenure
	}
} 
public class AbstractTest {
	public static void main(String[] args) {
		Employee e = new Salary("Sahil", 999999);
	}
}
\end{lstlisting}
\item \pt{1} \tf  \textit{e} is an instance of \textit{Employee}
\newline

\item \pt{1} \tf  You can directly instantiate an abstract class. e.g. e = new Employee("Varun");
\newline


\item \pt{1} \tf  An abstract class may contain non-abstract methods.
\newline

\item \pt{1} \tf  A method signature with the keyword abstract can not have a method body.
\newline

\item \pts{2} \tf  A \textsc{public class Manager extends Employee} is required to implement getName() \textbf{and} getTenure().
\newline

\item \pts{4}
If the Constructor for Employee was private, what would happen?

\begin{oneparchoices} 
\hspace{0.2cm}
 
 \choice  The behavior would stay the same since Salary extends Employee. \newline
 
 \choice The compiler would throw an error "Employee is abstract; cannot be instantiated" \newline
 
 \choice The compiler would throw an error "Employee(String) has private access in Employee"\newline
 
 \choice The behavior will stay the same, but the references being constructed would always be null. \newline
 \end{oneparchoices}




\vspace{2em}
\end{enumerate}
\end{document}
              