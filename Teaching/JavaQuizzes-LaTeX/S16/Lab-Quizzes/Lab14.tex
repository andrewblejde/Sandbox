\documentclass[addpoints]{exam}
\usepackage[utf8]{inputenc}
%%% Preamble for CS1800 Lab Quiz%%%
\usepackage[margin=1in]{geometry}
\geometry{letterpaper,textwidth=350pt,textheight=680pt,tmargin=60pt,
            left=72pt,footskip=24pt,headsep=18pt,headheight=14pt}
% Packages required to support encoding
\usepackage{ucs}
\usepackage[utf8x]{inputenc}
%%%%%%%%%%%%%%%%%%%
% To get dots between the entry name and the 
% page number in the Table Of Contents
\usepackage{tocloft}
\renewcommand\cftsecfont{\normalfont}
\renewcommand\cftsecpagefont{\normalfont}
\renewcommand{\cftsecleader}{\cftdotfill{\cftsecdotsep}}
\renewcommand\cftsecdotsep{\cftdot}
\renewcommand\cftsubsecdotsep{\cftdot}
%%%%%%%%%%%%%%%%%%%%
\usepackage{amsmath}
\usepackage{amssymb}
\usepackage[LGRgreek]{mathastext} % no italics in math mode
\usepackage{textcase}
\usepackage{soul}
\usepackage{booktabs} % for \midrule in eqnarray
\usepackage{indentfirst}
\usepackage{subfiles}
%%%%%%%%%%%%%%%%
\newif\ifdraft
%\drafttrue
\draftfalse
%%%%%%%%%%%%%%%%

\usepackage[pdftex]{graphicx}
\usepackage{hyperref}
\usepackage{xspace}
\usepackage[usenames,dvipsnames]{color}
\hypersetup{colorlinks=true,urlcolor=blue}
\usepackage{listings}
\definecolor{dkgreen}{rgb}{0,0.6,0}
\definecolor{gray}{rgb}{0.5,0.5,0.5}
\definecolor{mauve}{rgb}{0.58,0,0.82}

%%%%%%%%%%%%%%%%%%%%%%%%%%
% New command for inserting names in the document
\newcommand{\name}[1]{\textcolor{red}{\textbf{[#1]}}}
% New commands for inserting points in the document
\newcommand{\pt}[1]{\textbf{[#1 point]}}
\newcommand{\pts}[1]{\textbf{[#1 points]}}

\lstset{%frame=tb, % to form a frame around code
  language=Java,
  aboveskip=3mm,
  belowskip=3mm,
  showstringspaces=false,
  columns=flexible,
  basicstyle={\ttfamily}, % change to \small\ttfamily 
  numbers=none,
  numberstyle=\large\color{black},
  keywordstyle=\color{blue},
  commentstyle=\color{dkgreen},
  stringstyle=\color{mauve},
  breaklines=true,
  breakatwhitespace=true,
  tabsize=3
}
\usepackage[framemethod=TikZ]{mdframed}
\mdfdefinestyle{MyFrame}{%
    linecolor=black,
    outerlinewidth=1pt,
    roundcorner=20pt,
    innertopmargin=\baselineskip,
    innerbottommargin=\baselineskip,
    innerrightmargin=20pt,
    innerleftmargin=20pt,
    backgroundcolor=white!50!white}
%%%%%%%%%%%%%%%%

%% New Commands %%
\newcommand{\Ans}{\textcolor{red}{ [Answer] }}
\newcommand{\tf}{ (T / F) }
\newcommand{\Instructions}{Complete the following quiz questions and return the sheet to your TA. Do this at the start of the lab session.}
%%%%

\sodef\allcapsspacing{\upshape}{0.15em}{0.65em}{0.6em}%

\makeatletter
\def\maketitle{%
\par{\parskip 0pt
\hrule height 0.75pt\vspace{1ex}
\par\noindent
\begin{minipage}{0.6\textwidth}
\scshape
Purdue University -- CS 18000 \\
Problem Solving And \\ Object-Oriented Programming \end{minipage}%
\begin{minipage}{0.4\textwidth}
\raggedleft
\@title\\[0.2ex]
\textit{Name, Lab Sec: \hrulefill}\\[0.2ex]
\textit{\@date} %\textit{April 08, 2015} % 
\end{minipage}
\par\vspace{1ex}
\hrule height 1pt
\vspace{2ex}
\par
}}
\makeatother
\author{}
% auto generate a title
\AtBeginDocument{\maketitle}


\title{Lab 14 Quiz}
\date{Week 14}
\begin{document} 

\Instructions
\begin{enumerate}
\newline
\item \pts{2}
Which of these is \textbf{not} a dynamic data structure?

\begin{oneparchoices} 
\hspace{0.2cm}
 
 \choice Queue \newline
 
 \choice Array \newline
 
 \choice LinkedList\newline
 
 \choice Stack \newline
 \end{oneparchoices}
 
\item \pts{2} Complete the expressions for String a and String b, where ``a" stores the element at index 0 of List one, and ``b" stores the element at index 0 of List two.

\begin{lstlisting}
   List one = new ArrayList();
	one.add("Hello");
	
	List<String> two = new ArrayList<String>();
	two.add("World");

	String a =                          // TODO 1
	String b =                          // TODO 2

\end{lstlisting}

\item \pts{2}
Which of these is \textbf{not} true about a Stack?

\begin{oneparchoices} 
\hspace{0.2cm}
 
 \choice  A Stack could be implemented with a LinkedList. \newline
 
 \choice Stacks are first-in, first-out. \newline
 
 \choice New items can be added at the top of the Stack \newline
 
 \choice A Stack can store any object type. \newline
 \end{oneparchoices}

\item \pts{4}
Think about the DynamicBuffer class that you implemented in Project 4. Consider these functionalities for the DynamicBuffer;
\begin{lstlisting}
   void addEmail ( Email email ); 		// Add an email 
	boolean remove( int i ); 			    // Remove ith email
	Email[] getNewest( int n ); 		    // Get n most recent emails.    
\end{lstlisting}
From the dynamic data structures Queue, LinkedList, and HashMap which \textbf{one} would best fulfill these functionalities and why?
\newline
\newline

\vspace{2em}
\end{enumerate}
\end{document}
              