\documentclass[S17-Final.tex]{subfiles}
\begin{document}
\begin{enumerate}

\item What is the value of this expression: \texttt{true != true != true}

\begin{enumerate}
\item  \texttt{True}  
\item  \texttt{False}
\end{enumerate}

\item \texttt{new Object().equals(new Object())} is equivalent to \texttt{new Object() == new Object()}

\begin{enumerate}
\item  \texttt{True}  
\item  \texttt{False}
\end{enumerate}

\item An object of a generic class can be created with any type, \texttt{T}.

\begin{enumerate}
\item  \texttt{True}  
\item  \texttt{False}
\end{enumerate}

\item The following line would result in a compile-time error:
\begin{lstlisting}
int[][][][] array = new int[0][3][-2][1];
\end{lstlisting}

\begin{enumerate}
\item  \texttt{True}  
\item  \texttt{False}
\end{enumerate}

\item What is the 2’s complement of 23?

\begin{enumerate}
\item  \texttt{10010111}  
\item  \texttt{00010111}
\item  \texttt{00110111}  
\item  \texttt{01000011}
\item  None of the above
\end{enumerate}

\item How many layers of abstraction does Java have for file I/O?

\begin{enumerate}
\item  1
\item  2
\item  3
\item  There’s no specific number
\item  None of the above
\end{enumerate}

\clearpage

\item Which of the following is NOT a valid overloaded method for the method below?
\begin{lstlisting}
public static int max (int x, int y) {}
\end{lstlisting}

\begin{enumerate}
\item  \texttt{public static int max (int x) {}}  
\item  \texttt{public static int max (int x, int y, double z) {}}
\item  \texttt{public static double max (int x, int y) {}}  
\item  \texttt{public static double max (double x, double y) {}}
\item  \texttt{public static String max (String x, String y) {}}  
\end{enumerate}

\item Consider the following program: 
\begin{lstlisting}
public class Runner implements Runnable { 
  private String name;

    public Runner(String name) { 
      this.name = name; 
    } 

    public void run() { 
      System.out.print(name); 
    } 

  public static void main(String[] args) { 
    Thread quick = new Thread(new Runner("Quick")); 
    Thread slow = new Thread(new Runner("Slow")); 
    quick.start(); 
    slow.start(); 
  } 
} 
\end{lstlisting}

Which of the following does it produce as output? 

\begin{enumerate}
\item  Quick
\item  Slow
\item  Slow is repeatedly printed as the program executes an infinite loop 
\item  Nothing is printed 
\item  Either of QuickSlow or SlowQuick
\end{enumerate}

\item Which of the following assertions about method overriding is true? 

\begin{enumerate}
\item  The signature of the overridden method must be different from the signature of the original method 
\item  Overriding is the same as overloading 
\item  A derived class can override a method of the base class
\item  A method can be overridden only in the same class where the original method is defined 
\item  None of the above assertions is true
\end{enumerate}

\item What is the output of the following code?
\begin{lstlisting}
public class MyClass {
  public static void main(String[] args) {
    int z = 102;
    int y = 3;

    if (z < 100 && (y = 10) < 5)
     z++; 

    System.out.printf("%d, %d\n", y, z);
  }
}
\end{lstlisting}

\begin{enumerate}
\item  10, 102 
\item  3, 102 
\item  5, 102 
\item  10, 103 
\item  3, 103
\end{enumerate}

\item What is the output of the following code?
\begin{lstlisting}
   String str = "Hello, world!";
	str.replaceAll("l", "1");
	System.out.println(str.substring(str.indexOf(",")));
\end{lstlisting}

\begin{enumerate}
\item  "Hello, world!"
\item  ", wor1d!"
\item  ", world!"
\item  "He110o, wor1d!"
\item  "He11o"
\end{enumerate}

\item A class, \texttt{Tire}, extends \texttt{Wheel}. Which of the following is true?

\begin{enumerate}
\item  Wheel cannot have subclasses other than Tire.
\item  Tire cannot have any subclasses.
\item  Tire cannot have superclasses other than Wheel.
\item  Wheel cannot have any superclasses.
\item  Java doesn’t follow ‘single inheritance’.
\end{enumerate}

\clearpage 

\item Consider the following code:
\begin{lstlisting}
public class Point {
  private int x;
  private int y;

  public Point() {
    this.x = 0; this.y = 0;
  } //Point
}
public class PointRunner {
  public static void main(String[] args) {
    Point p = new Point();

    p.x = 2; p.y = 2;

    System.out.printf("(%d, %d)\n", p.x, p.y);
  } //main
}
\end{lstlisting}

What is the output after running main()?

\begin{enumerate}
\item  "(0, 0)"
\item  "(2, 2)"
\item  "(null, null)"
\item  Compile-time error
\item  Runtime error
\end{enumerate}

\item What is a class?

\begin{enumerate}
\item  an object
\item  a data structure
\item  a blueprint for an object
\item  a singleton
\item  an enumeration
\end{enumerate}

\item What is the output of the following code?
\begin{lstlisting}
  Integer iOne = 5;
  Integer iTwo = new Integer(5);
  if (iOne == iTwo) {
    System.out.printf("%010d", iOne + iTwo);
  } else {
    System.out.printf("%05d", iOne * iTwo);
  }
\end{lstlisting}
\clearpage
\begin{enumerate}
\item  0000000010
\item  00025
\item  01010
\item  0525
\item  000010001
\end{enumerate}


\end{enumerate}
\end{document}
