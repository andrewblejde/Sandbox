\Instructions
   Explain your reasoning to obtain partial credit.
\begin{enumerate}
    
\item \pts{2} \tf The element at index 1 of arr is A.
\begin{lstlisting}
char[] arr = {'A' , 'B' , 'C' , 'D'};
\end{lstlisting}

\item \pts{2} Complete the code below for a 2-dimensional jagged array \texttt{arr}. A jagged array is an array whose elements are arrays. The elements of a jagged array can be of different dimensions and sizes. 
\begin{lstlisting}
for(int i = 0; i < arr.length; i++) {
	for( /* your code here */ ) {
    	...
    }
}
\end{lstlisting}
 
\item \pts{2} \tf Using the array from Q1 as an example, the last element of the array can be retrieved to a variable \texttt{lastElement} of type char by the statement
\begin{lstlisting}
char lastElement = arr[arr.length];
\end{lstlisting}

\item \pts{4} Given the following snippet of code, write the output printed to stdout. 

\begin{lstlisting}
int[][] arr = {{1/8, 2/8, 3/8},
			   {4/8, 5/8},
			   {6/8}};

for(int i = 0; i < arr.length; i++) {
	for(int j = 0; j < arr[i].length; j++)
		System.out.print(arr[i][j] + " ");
	System.out.println();
}
\end{lstlisting}

\clearpage

\clearpage
\ifdraft \clearpage \fi

\end{enumerate}   
\end{document}
              
