\Instructions 
  Explain your reasoning to obtain partial credit.
\begin{enumerate}

Consider the following code for questions 1, 2, 3 and 4
\begin{lstlisting}
class A {
	public A(String msg) { }
}

class B extends A {
	public B() {
		this("Hello World"); /* #1 */
	}

	public B(String str) {
		super(str); /* #2 */
	}
}

class C extends B {
}

class Main {
	public static void main(String[] args) {
		A ab = new B(); /* #3 */
		C c = new C(); /* #4 */
	}
}
\end{lstlisting}

\item \pts{2} \tf \#1: calls the superclass Constructor of A immediately. \Ans False

\vspace{2em}
\item \pts{2} \tf \#2: calls the superclass Constructor of A immediately. \Ans True

\vspace{2em}
\item \pts{2} \tf \#3: \texttt{ab instanceof A \&\& ab instanceof B} evaluates to true. \Ans True

\vspace{2em}
\item \pts{2} What is the proper constructor chain for \#4?
\begin{enumerate}
\item Will go straight to \texttt{B()}
\item Will go straight to \texttt{B()} then to \texttt{B(String str)} then to \texttt{A(String msg)}
\item Will go to default constructor \texttt{C()} then to \texttt{B()} then to \texttt{B(String str)} then to \texttt{A(String msg)}
\item Will go straight to \texttt{A(String msg)}\\
\Ans C
\end{enumerate}

\vspace{2em}
\item \pts{2} \tf A superclass constructor can call a subclass constructor. \Ans False




\end{enumerate}  